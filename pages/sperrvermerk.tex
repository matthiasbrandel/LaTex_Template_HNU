\chapter*{Sperrvermerk}
\addcontentsline{toc}{chapter}{Sperrvermerk}
\enlargethispage{3pt}
\linespread{1}
\WarningsOff[nag]
\noindent
Die vorliegende \getThesisType{} von \getAuthor{} mit dem Thema „\getTitle{}“ enthält vertrauliche Daten. Sie unterliegt einer Einsichtssperre für Personen, die nicht mit der Korrektur der Arbeit beauftragt sind. Außerdem besteht über den Inhalt der Arbeit Stillschweigen. 

\noindent
Der Sperrvermerk wird auf einen Zeitraum von 5 Jahren ab Abgabe der Masterarbeit begrenzt.

\noindent
Folgende Vorgänge sind jedoch gestattet (bitte ankreuzen):

\vspace{5mm}
\noindent\textit{Veröffentlichung von Daten in Publikations-Datenbanken:}
\par\noindent$\square$ Titel der Arbeit mit Unternehmensnamen
\par\noindent$\square$ Kurztitel ohne Unternehmensnamen 
\par\noindent$\square$ Abstract 
\vspace{5mm}
\par\noindent\textit{Im Rahmen von Preisverleihungen:} 
\begin{table}[h]
\footnotesize 
\noindent
\begin{tabularx}{15cm}{|X|X|}
\hline
\multicolumn{2}{|>{\hsize=\dimexpr2\hsize+2\tabcolsep+\arrayrulewidth\relax}X|}{$\square$ Weitergabe von Titel, wissenschaftlicher Würdigung und Abstract der Arbeit an ein
internes Gremium der HNU zur Prüfung der Preiswürdigkeit
(ohne diese Zustimmung kann die Arbeit bei Preisverleihungen nicht berücksichtigt werden!)
}                                                               \\ \hline
Weitergabe von: 
\par\noindent$\square$ Titel mit Unternehmensnamen (ggf.)
\par\noindent$\square$ wissenschaftliche Würdigung
\par\noindent$\square$ Abstract der Arbeit
& 
an:
\par\noindent$\square$ eine externe Jury / den Preisstifter
\par\noindent$\square$ Vertreter der Presse\\
\hline
\end{tabularx}
\end{table}

\begin{table}[h]
\footnotesize 
\begin{tabularx}{15cm}{|X|}
\hline
Hinweis: Die bibliographischen Angaben der Abschlussarbeit (Autor, Titel, betreuende/r Professor/in u.a.) werden nicht vom Sperrvermerk umfasst und werden in der Datenbank „Abschlussarbeiten“ auf den Internetseiten der Hochschulbibliothek veröffentlicht. Die Abschlussarbeit kann im automatisierten Verfahren durch eine Plagiatssoftware geprüft werden.\\
\hline
\end{tabularx}
\end{table}
\vfill
\begin{table}[H]
\begin{tabularx}{15cm}{XXX}
\cline{1-1} \cline{3-3}
\centering
\getAuthor{} &  & \getCompany{}
\end{tabularx}
\end{table}
\WarningsOn[]